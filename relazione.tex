\documentclass[12pt, a4paper]{article} % carattere 12 (possiamo anche fare 11)
\usepackage{graphicx} % da vedere se tenerlo
\title{Simulazione dell'evoluzione di una pandemia sulla base del modello SIR}
\author{Matilda Pasquini e Silvia Vicentini}
\begin{document}
\maketitle
\section{Abstract}

\section{Cenni teorici}
Il programma \`{e} stato sviluppato col fine di simulare la diffuzione di un'epidemia sulla base del modello teorico SIR. Data una popolazione chiusa di $N$ abitandi, questi vengono suddivisi in
\begin{itemize}
\item I \textit{Suscettibili}, cio\`{e} coloro che possono essere contagiati
\item Gli \textit{Infetti}, cio\`{e} coloro che possono trasmettere la malattia
\item I \textit{Rimossi}, cio\`{e} coloro che sono guariti o morti
\end{itemize}
Dato che le morti per maattia sono incluse nella categoria R e le nascite o morti per altre cause non vengono tenute in considerazione, vale la relazione $S+R+I=N$. Lo stato di una persona pu\`{o} evolvere in una sola direzione, prima da suscettibile a infetto e poi da infetto a rimosso. Questo perch\'e ogni individuo pu\`{o} essere infettato dall'agente patogeno non pi\`{u} di una volta, per poi diventare immune o morire. 
L'evoluzione dell'epidemia dipende da due fattori:
\begin{itemize}
\item \textit{$\beta$} \`{e} il tasso di contagio, ovvero esprime la probabilit\`{a} di essere contagiati

\item \textit{$\gamma$} \`{e} il tasso di guarigione, ovvero la probabilit\`{a} di guarire
\end{itemize}

Entrambi i parametri devono assumere un valore compreso tra 0 e 1. Inoltre, nel modello teorico, beta e gamma sono considerati costanti nonostante nella realtà siano tipicamente variabili poichè dipendono dalle misure di mitigazione messe in atto.

La variazione del numero di individui all'interno di ciascuno dei tre raggruppamenti segue le leggi differenziali
\begin{equation}
\frac{dS}{dt} = -\beta \frac{S}{N}I \\
\frac{dI}{dt} = \beta \frac{S}{N}I-\gamma I  \\ % bisogna fare in modo che siano su righe diverse allineate
\frac{dR}{dt} = \gamma I \\
\end{equation}

Discretizzando le equazioni differenziali sopra citate e considerando $$\Delta$$T = 1 si ottengono le seguenti equazioni:
\begin{equation}
\$$S_i$$ = $$S_i-1$$


\section{Struttura del programma}

\section{Compilare}
\section{Testare}  % indicare la strategia di test
\section{Eseguire} % inserire un esempio dei risultati 
% parlare dei tre input e del grafico
\section{Conclusione}

\subsection{First Subsection} %potrebbero tornare utili


\end{document} 
