\documentclass[12pt, a4paper]{article} % carattere 12 (possiamo anche fare 11)
\usepackage{graphicx} % da vedere se tenerlo
\title{Simulazione dell'evoluzione di una pandemia sulla base del modello SIR}
\author{Matilda Pasquini e Silvia Vicentini}
\begin{document}
\maketitle
\section{Abstract}

\section{Cenni teorici}
Il programma \è stato sviluppato col fine di simulare la diffuzione di un'epidemia sulla base del modello teorico SIR. Data una popolazione chiusa di $N$ abitandi, questi vengono suddivisi in
\begin{itemize}
\item I \textit{Suscettibili}, cio\è coloro che possono essere contagiati
\item Gli \textit{Infetti}, cio\è coloro che possono trasmettere la malattia
\item I \textit{Rimossi}, cio\è coloro che sono guariti o morti
\end{itemize}
Dato che le morti per maattia sono incluse nella categoria R e le nascite o morti per altre cause non vengono tenute in considerazione, vale la relazione $S+R+I=N$. Chi fa parte della categoria S pu\`{o} solo spostarsi nella I, parallelamente chi nella I pu\`{o} solamente muoversi al gruppo R. Questo perch\'e ogni individuo pu\`{o} essere infettato dall'agente patogeno non pi\`{u} di una volta, per poi diventare immune o morire. 
Lo spargimento dell'agente patogeno dipende da due fattori:
\begin{itemize}
\item \textit{\beta} \`{e} il tasso di contagio, ovvero esprime la probabilit\`{a} di essere contagiati.
\item \textit{\gamma} \`{e} il tasso di guarigione, ovvero la probabilit\`{a} di guarire.
\end{itemize}
La variazione del numero di individui all'interno di ciascuno dei tre raggruppamenti segue le leggi differenziali
\begin{equation}

\end{equation}

Bisogna evidenziare che il modello seguito apporta notevoli semplificazioni alla reale diffusione di un'epidemia, ma consente comunque di studiarne il suo andamento generale.

\section{Struttura del programma}

\section{Compilare}
\section{Testare}  % indicare la strategia di test
\section{Eseguire} % inserire un esempio dei risultati 
% parlare dei tre input e del grafico
\section{Conclusione}

\subsection{First Subsection} %potrebbero tornare utili


\end{document} 
