\documentclass[11pt, a4paper]{article} % carattere 11
\usepackage{graphicx} % da vedere se tenerlo
\title{Simulazione dell'evoluzione di una pandemia sulla base del modello SIR}
\author{Matilda Pasquini e Silvia Vicentini}
\begin{document}
\maketitle

\section{Cenni teorici}
Il programma \`{e} stato sviluppato col fine di simulare la diffuzione di 
un'epidemia sulla base del modello teorico SIR. Data una popolazione chiusa di 
$N$ abitandi, questi vengono suddivisi in
\begin{itemize}
\item $S$, i \textit{Suscettibili}, cio\`{e} coloro che possono essere contagiati
\item $I$, gli \textit{Infetti}, cio\`{e} coloro che possono trasmettere la malattia
\item $R$, i \textit{Rimossi}, cio\`{e} coloro che sono guariti o morti
\end{itemize}
Vale dunque la relazione $S+R+I=N$ dal momento che le morti per malattia sono 
incluse nella categoria R e le nascite o le morti per altre cause non vengono 
tenute in considerazione. Chi fa parte della categoria S pu\`{o} solo spostarsi 
nella I, parallelamente, chi nella I pu\`{o} solamente muoversi nel gruppo R. 
Questo perch\'e ogni individuo pu\`{o} essere infettato dall'agente patogeno 
non pi\`{u} di una volta, per poi diventare immune o morire. Dato che i valori 
$S$, $I$, $R$ rappresentano il numero di abitanti per ciascun insieme 
\[S, I, R \in \mathbb{N}\]
Lo spargimento dell'agente patogeno dipende da due fattori:
\begin{itemize}
\item \textit{$\beta$}, il tasso di contagio, ovvero la probabilit\`{a} di
essere contagiati
\item \textit{$\gamma$}, il tasso di guarigione, ovvero la probabilit\`{a} di 
guarire
\end{itemize}
La variazione del numero di individui all'interno di ciascuno dei tre 
raggruppamenti \`{e} definita dalle seguenti leggi differenziali
\begin{equation} 
\frac{dS}{dt} &= -\beta\frac{S}{N}I  \\
\frac{dI}{dt} &= \beta\frac{S}{N}I-\gamma I \\
\frac{dR}{dt} &= \gamma I 
\end{equation}
L'epidemia \`{e} in espansione a condizione che $\frac{dI}{dt} > 0$. 
Dato che \[\frac{dI}{dt}=(\beta\frac{S}{N}-\gamma) I>0\] 
allora $\frac{\beta}{\gamma}>\frac{N}{S}$. Poich\`{e} all'inizio dell'epidemia 
$S\approx N$, la malattia si diffonde solo se \`{e} verificata la condizione 
che il rapporto $R_0=\frac{\beta}{\gamma}>1$.\\
Discretizzando le tre leggi e studiando la variazione di individui in ciascun 
gruppo in un arco temporale $\Delta t=1$ giorno, si ottengono le seguenti formule
\begin{equation}  %bisogna allinearle, come?
S_i = S_{i-1} - \beta \frac{S_{i-1}}{N} I_{i-1} \\
I_i = I_{i-1} + \beta \frac{S_{i-1}}{N} I_{i-1} - \gamma I_{i-1} \\
R_i = R_{i-1} + \gamma I_{i-1} 
\end{equation}
Il progetto si sviluppa a partire da queste equazioni.
Tale modello teorico apporta notevoli semplificazioni alla reale 
diffusione di un'epidemia, nonostante ci\`{o}, esso consente di trattare 
l'andamento generale del problema in maniera semplice e chiara.

\section{Struttura del programma}
Il progetto \`{e} organizzato nei seguenti files 
\begin{itemize}
\item \textit{epidemic.hpp}, un header file, contenente le definizioni di classe 
e le dichiarazioni delle finzioni membro 
\item \textit{epidemic.cpp}, un source file, dove si trovano le definizioni delle
 funzioni membro
\item \textit{epidemic.test.cpp}, l'unit test, che consente di eseguire i test
\item \textit{main.cpp}, per testare direttamente l'implementazione del programma
\item \textit{CMake}, per semplificare la compilazione del progetto dato che è 
costituito da pi\`{u} di un file sorgente
\item \textit{doctest.h}, necessario per poter eseguire i test scritti in 
\textit{epidemic.test.cpp}
\item \textit{.clang-format}, per formattare adeguatamente il programma
  %forse dovrò aggiungere un file per la libreria grafica

\subsection{epidemic.hpp}


\section{Conclusione}


\end{document} 
