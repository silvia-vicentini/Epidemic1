\documentclass[12pt, a4paper]{article} % carattere 12 (possiamo anche fare 11)
\usepackage{graphicx} % da vedere se tenerlo
\title{Simulazione dell'evoluzione di una pandemia sulla base del modello SIR}
\author{Matilda Pasquini e Silvia Vicentini}
\begin{document}
\maketitle
\section{Abstract}

\section{Cenni teorici}
Il programma \`{e} stato sviluppato col fine di simulare la diffuzione di un'epidemia sulla base del modello teorico SIR. Data una popolazione chiusa di $N$ abitandi, questi vengono suddivisi in
\begin{itemize}
\item I \textit{Suscettibili}, cio\`{e} coloro che possono essere contagiati
\item Gli \textit{Infetti}, cio\`{e} coloro che possono trasmettere la malattia
\item I \textit{Rimossi}, cio\`{e} coloro che sono guariti o morti
\end{itemize}
Dato che le morti per maattia sono incluse nella categoria R e le nascite o morti per altre cause non vengono tenute in considerazione, vale la relazione $S+R+I=N$. Lo stato di una persona pu\`{o} evolvere in una sola direzione, prima da suscettibile a infetto e poi da infetto a rimosso. Questo perch\'e ogni individuo pu\`{o} essere infettato dall'agente patogeno non pi\`{u} di una volta, per poi diventare immune o morire. 
L'evoluzione dell'epidemia dipende da due fattori:
\begin{itemize}
\item \textit{$\beta$} \`{e} il tasso di contagio, ovvero esprime la probabilit\`{a} di essere contagiati

\item \textit{$\gamma$} \`{e} il tasso di guarigione, ovvero la probabilit\`{a} di guarire
\end{itemize}

Entrambi i parametri devono assumere un valore compreso tra 0 e 1. Inoltre, nel modello teorico, beta e gamma sono considerati costanti nonostante nella realtà siano tipicamente variabili poichè dipendono dalle misure di mitigazione messe in atto.

La variazione del numero di individui all'interno di ciascuno dei tre raggruppamenti segue le leggi differenziali
\begin{equation}
\frac{dS}{dt} = -\beta \frac{S}{N}I \\
\frac{dI}{dt} = \beta \frac{S}{N}I-\gamma I  \\ % bisogna fare in modo che siano su righe diverse allineate
\frac{dR}{dt} = \gamma I \\
\end{equation}

Discretizzando le equazioni differenziali sopra citate e considerando $$\Delta$$T = 1 si ottengono le seguenti equazioni:
\begin{equation}
\$$S_i$$ = $$S_i-1$$


\section{Struttura del programma}
Il progetto \`{e} organizzato nei seguenti files 
\begin{itemize}
\item \textit{epidemic.hpp}, un header file, contenente le definizioni di classe 
e le dichiarazioni delle finzioni membro 
\item \textit{epidemic.cpp}, un source file, dove si trovano le definizioni delle
 funzioni membro
\item \textit{epidemic.test.cpp}, l'unit test, che consente di eseguire i test
\item \textit{main.cpp}, per testare direttamente l'implementazione del programma
\item \textit{CMake}, per semplificare la compilazione del progetto dato che è 
costituito da pi\`{u} di un file sorgente
\item \textit{doctest.h}, necessario per poter eseguire i test scritti in 
\textit{epidemic.test.cpp}
\item \textit{.clang-format}, per formattare adeguatamente il programma
  %forse dovrò aggiungere un file per la libreria grafica

\subsection{epidemic.hpp}
Il file epidemic.hpp è un harder file contentente le definizioni delle classi e le dichiarazioni delle funzioni membro.
Inanzitutto viene definita una struttura chiamata "Population", la quale rappresenta una popolazione suddivisa in tre gruppi: S,I e R. Ognuno di questi tre gruppi è rappresentata da una variabile intera all'interno della struttura.
Viene inoltre definita la classe Epidemic, all'interno della quale vengono definite nella parte private:
\item \textit{double const beta_}, una variabile membro costante di tipo double, chiamata "beta_" 
\item \textit{double const gamma_}, una variabile membro costante di tipo double, chiamata "gamma_"
\item \textit{std::vector<Population> population_state}, un vettore di oggetti di tipo "Population"
\item \textit{Population solve(Population const, int const) const}, una funzione membro della classe.
La definizione della calsse Epidemic contiene inoltre anche dei metodi pubblici, che sono i seguenti:
\item \textit{Epidemic(double const, double const)}, cioè il costruttore della classe che prende come argomento due parametri di tipo double e viene utilizzato per inizializzare le variabili private della classe beta_ e gamma_
\item \textit{std::vector<Population> evolve(Population, int const)} membro chiamato "evolve" che prende come argomenti un oggetto population e un intero costante e che restituisce un vettore di oggetti "Population" i quali rappresentano lo stato evoluto della popolazione nel tempo.
\item \textit{std::vector<Population> const &state() const}




\section{Conclusione}

\section{Compilare}
\section{Testare}  % indicare la strategia di test
\section{Eseguire} % inserire un esempio dei risultati 
% parlare dei tre input e del grafico
\section{Conclusione}

\subsection{First Subsection} %potrebbero tornare utili


\end{document} 
