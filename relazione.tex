\documentclass[12pt, a4paper]{article} % carattere 12 (possiamo anche fare 11)
\usepackage{graphicx} % da vedere se tenerlo
\title{Simulazione dell'evoluzione di una pandemia sulla base del modello SIR}
\author{Matilda Pasquini e Silvia Vicentini}
\begin{document}
\maketitle
\section{Abstract}

\section{Cenni teorici}
Il programma \`{e} stato sviluppato col fine di simulare la diffuzione di un'epidemia sulla base del modello teorico SIR. Data una popolazione chiusa di $N$ abitandi, questi vengono suddivisi in tre categorie:
\begin{itemize}
\item I \textit{Suscettibili}, cio\`{e} coloro che possono essere contagiati.
\item Gli \textit{Infetti}, cio\`{e} coloro che possono trasmettere la malattia.
\item I \textit{Rimossi}, cio\`{e} coloro che si sono già ammalati e sono guariti o morti e che quindi non possono più traspettere la malattia.
\end{itemize}

\section{Struttura del programma}

\section{Compilare}
\section{Testare}  % indicare la strategia di test
\section{Eseguire} % inserire un esempio dei risultati 
% parlare dei tre input e del grafico
\section{Conclusione}

\subsection{First Subsection} %potrebbero tornare utili


\end{document} 
